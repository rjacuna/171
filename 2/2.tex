\documentclass{article}
\usepackage{fontspec}
\usepackage{xcolor}
\usepackage{sagetex}

\usepackage{euler}
\usepackage{amsmath}
\usepackage{amssymb}
\usepackage{unicode-math}


\usepackage[makeroom]{cancel}
\usepackage{ulem}

\setlength\parindent{0em}
\setlength\parskip{0.618em}
\usepackage[a4paper,lmargin=1in,rmargin=1in,tmargin=1in,bmargin=1in]{geometry}

\setmainfont[Mapping=tex-text]{Helvetica Neue LT Std 45 Light}

\newcommand\N{\mathbb{N}}
\newcommand\Z{\mathbb{Z}}
\newcommand\R{\mathbb{R}}
\newcommand\C{\mathbb{C}}
\newcommand\A{\mathbb{A}}

\usepackage{soul}
\begin{document}

\begin{center}
  171 --- Homework 2

  Ricardo J. Acuna

  (862079740)
\end{center}\vspace{1.618em}

\subsection*{Section 3}

\paragraph{2}

\paragraph{4}

\paragraph{6}

\paragraph{8}

\paragraph{18}

\subsection*{Section 4}

\paragraph{4}

\paragraph{8}

\paragraph{9}

\paragraph{31}

\paragraph{34}

\subsection*{Aditional Excercises}

\paragraph{1} Let $G$ be a group and $a \in G$. Prove that $(a^\prime)^\prime
= a.$

\paragraph{2} Let $\langle S,* \rangle$ be a binary algebraic
structure and define Aut$(S)$ be the set of isomorphism from $S$ to
$S$. That is,

\[\text{Aut}(S) = \{f:S\rightarrow S| f \text{ is an automorphism}\} \]

Prove that $\lange$ Aut$(S) , \circ \rangle$ is a group, where $\circ$
is the usual function composition. You don't need to prove that
$\circ$ is associative (this is well-known) and you may use results
used in class as long as you cite them.



\end{document}


%%% Local Variables:
%%% mode: latex
%%% TeX-master: t
%%% End:
