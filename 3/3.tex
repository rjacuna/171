\documentclass{article}
\usepackage{fontspec}
\usepackage{xcolor}
\usepackage{sagetex}

\usepackage{euler}
\usepackage{amsmath}
\usepackage{amssymb}
\usepackage{unicode-math}


\usepackage[makeroom]{cancel}
\usepackage{ulem}

\setlength\parindent{0em}
\setlength\parskip{0.618em}
\usepackage[a4paper,lmargin=1in,rmargin=1in,tmargin=1in,bmargin=1in]{geometry}

\setmainfont[Mapping=tex-text]{Helvetica Neue LT Std 45 Light}

\newcommand\N{\mathbb{N}}
\newcommand\Z{\mathbb{Z}}
\newcommand\R{\mathbb{R}}
\newcommand\C{\mathbb{C}}
\newcommand\A{\mathbb{A}}

\begin{document}

\begin{center}
  171 --- 3

  RJ Acuna

  (862079740)
\end{center}\vspace{1.618em}

\subsection*{Section 4}

\paragraph{40} Let $\langle {G,\cdot} \rangle$ be a group. Consider
the binary operation $*$ on the set $G$ defined by

\[a*b = b\cdot a\]

for $a,b\in G$. Show that $\langle {G,*} \rangle$ is a group, and that
$\langle {G,*} \rangle$ is actually isomorphic to $\langle {G,\cdot}
\rangle$. [Hint: consider the map $\phi$ with $\phi(a) = a^\prime$ for
$a\in G$]

\uwave{pf.}

By hint, consider $\phi:G\rightarrow G; a\mapsto a^\prime$.

Suppose $\exists a,b\in G: \phi(a)=\phi(b)$
\[ \phi(a) =\phi(b) \implies
   a^\prime = b^\prime\implies
   a^\prime* b = b^\prime* b\implies
   b \cdot a^\prime = b\cdot b^\prime \implies
   b \cdot a^\prime = e \implies
   b = (a^{\prime})^{\prime} \implies
 b = a\enskip.\]

 So, $\phi$ is injective.

 Let $c \in G$, we want to show $\exists x\in G: \phi(x)= c$,

Since $G$ is a group $c^\prime \in G,$ then $\phi(c^\prime) =
(c^\prime)^\prime$
\[ \phi(c^\prime) =
(c^\prime)^\prime \implies
   \phi(c^\prime)*c^\prime =
   (c^\prime)^\prime*c^\prime \implies \phi(c^\prime)*c^\prime = e
   \implies \phi(c^\prime) = c\enskip.
 \]

 So, $\phi$ is surjective. So,$\phi$ is a bijection.

 We want to show $\forall x,y \in G: \phi(x\cdot y) = x*y$.
 \[ \phi(x\cdot y) = (x \cdot y)^\prime = y^\prime\cdot x^\prime =
   \phi(y)\cdot \phi(x) = \phi(x)*\phi(y)\enskip.
 \]

So $\langle {G,\cdot} \rangle$ and $\langle {G,*} \rangle$ are
isomorphic binary structures.

 Now to show $\langle {G,*} \rangle$ is a group note:

Let $z\in G$, then compute $z*e = e\cdot z = z = z\cdot e = e*z$. So,
$e$ is the identity element of $\langle {G,*} \rangle$. This follows
from $\langle {G,*} \rangle$ being a binary structure, if it has an
identity element it must be unique.

Let $a,b,c\in G$, then compute
\[(a*b)*c = (b\cdot a)*c = c\cdot(a\cdot b) = (c\cdot a)\cdot b =
  (a*c)\cdot b = b*(a*c) \enskip.\]

Which holds since $\langle {G,\cdot} \rangle$ is a group.

Let $w\in G$, then compute

\[w*w^\prime = w^\prime \cdot w = e\]

So, $w^\prime$ is the inverse of $w$ with respect to $*$, and it
exists and is in $G$, since $\langle {G,\cdot} \rangle$ is a group.

So, $\langle {G,*} \rangle$ is a group, and $\langle {G,\cdot}
\rangle\simeq \langle {G,*} \rangle$
$\blacksquare$

\uwave{alt. pf.} Note, $\phi\circ \phi = 1_G$, so $\phi$ is a
bijection. Prove the homomorphism property as above. So $\phi$ is
an isomorphism. Then note $\langle {G,*} \rangle$ is a group since
isomorphisms preserve the algebraic structure. $\blacksquare$

\subsection*{Section 5}

In exercises $8$ through $13$, determine whether the given set of
invertible $n\times n$ matrices with real number entries is a subgroup
of $GL(n,\R).$

\paragraph{8} The $n\times n$ matrices with determinant $2$

\uwave{slu.}

No, $\begin{pmatrix}1&1\\ 0&2\end{pmatrix}$ and $\begin{pmatrix}2&1\\
  0&1\end{pmatrix}$ have determinant $2$.

But, $\begin{pmatrix}1&1\\ 0&2\end{pmatrix}\begin{pmatrix}2&1\\
  0&1\end{pmatrix} = \begin{pmatrix}2&2\\
  0&2\end{pmatrix}$, has determinant $4$. So, the $n\times n$ matrices
with determinant $2$ is not closed under matrix multiplication$\enskip
\lozenge$

\paragraph{13} The set of all $n\times n$ matrices $A$ such that $A^T
A = I_n.$ [These matrices are called \textbf{orthogonal}. Recall $A^T$,
is called the \textit{transpose} of $A$, is the matrix whose $j$th
column is the $j$th row of $A$ for $1\leq j\leq n$, and that the
transpose operation has the property $(AB)^T = B^TA^T$.]

\uwave{pf.}

$O(n):= \{X\in GL(n,\R)| X^TX=I_n\}$, denote $I := I_n$.

Let $A,B \in O(n)$
\[(AB)^TAB = B^TA^TAB =B^TIB = B^TB = I \implies AB\in O(n)\enskip.\]

So, $O(n)$ is closed under multiplication.
\[I^TI = (II^T)^T = (I^T)^T = I\enskip.\]
So, $I\in O(n)$.
\[\forall X \in O(n),\quad XI = X = IX\enskip.\]
So,  $I$ is the identity element of $O(n)$.

The associativity of $O(n)$ is inherited from $GL(n,\R)$.

Since $\forall A\in O(n), A^TA = I$, it follows that $A^T =
A^{-1}\quad \square$

In exercises $22$ through $25$, describe all the elements in
$GL(2,\R)$ generated by the given $2\times 2$ matrix.

\begin{sagesilent}
  from sympy import *
  init_printing(use_unicode=True)
  A = Matrix([[0,-1],[-1,0]])
  B = Matrix([[3,0],[0,2]])
  C = Matrix([[1,1],[0,1]])
  t = [latex(C^n) for n in [-3..3]]
  s = [latex(B^n) for n in [-3..3]]
\end{sagesilent}

\paragraph{22} $\sage{A}$

\uwave{slu.}

$\sage{A}\sage{A} = \sage{A*A}$, so $\langle {\sage{A}} \rangle = \{I,\sage{A}\}\quad \blacklozenge$

\paragraph{23}

\uwave{slu.}

Let $M_{23} = \sage{C}$
\begin{align*}
\langle {M_{23}}\rangle &= \left\{\dots, \sagestr{", ".join(t)}, \dots
                            \right\}\\
                          &= \left\{\begin{bmatrix}
                            1&n\\0&1\end{bmatrix}| n\in \Z\right\}\quad \blacklozenge
\end{align*}


\paragraph{24} $\sage{B}$

I'm a fucking idiot. So I did this one and I didn't have to.
\uwave{slu.}

Let $M_{24} = \sage{B}$
\begin{align*}
\langle {M_{24}}\rangle &= \left\{\dots, \sagestr{", ".join(s)}, \dots
                            \right\}\\
                          &= \left\{\begin{bmatrix}
                            3^n&0\\0&2^n\end{bmatrix}| n\in \Z\right\}\quad \blacklozenge
\end{align*}

In Exercises $27$ through $35$, find the order of the cyclic subgroup
of the given subgroup generated by the indicated element.

\paragraph{35} The subgroup of the multiplicative group $G$ of
invertible $4\times 4$ matrices generated by

\begin{sagesilent}
  from sympy import *
  init_printing(use_unicode=True)
  M35 = Matrix([[0,1,0,0],[0,0,0,1],[0,0,1,0],[1,0,0,0]])
  r = [latex(M35^n) for n in [1..3]]
\end{sagesilent}

Let $M_{35} = \sage{M35}$
\begin{align*}
  \langle {M_{35}}\rangle &=
                            \left\{M_{35},M_{35}^2,M_{35}^3,M_{35}^4,\dots\right\}\\
                        &= \left\{\sagestr{", ".join(r)}\right\}
\end{align*}
\[\implies |\langle {M_{35}}\rangle| = 3\quad \blacklozenge\]

\paragraph{45} Show that a non-empty subset $H$ of a group $G$ if and
only if $ab^{-1}\in H$ for all $a,b \in H$. (This is one of the
\textit{ more compact criteria } referred to prior Theorem 5.14)

\uwave{pf.}

Let $H\subset G$ and $H\neq \emptyset$. Let $a,b \in H$.

($\implies$) Ass. $H\leq G$.

$H\leq G \implies b^{-1}\in H$ and $H$ is closed $\implies ab^{-1}\in H$.

($\impliedby$) Ass. $\forall a,b \in H,$ $ab^{-1}\in H$.

$\forall a,b,c \in H, a(bc) = (ab)c$, since it is also an equation in
$G$.

$\forall a,b \in H \implies ab^{-1}\in H$ and $ba^{-1} \in H$

$\implies  aa^{-1} = e = a^{-1}a \in H$. So the identity is in $H$.

$\implies a^{-1}a(ba^{-1})^{-1} = a^{-1}ab^{-1}a = b^{-1}a \in H$

$\implies b^{-1}aa^{-1} = b^{-1}e= b^{-1} \in H$

Similarly, $a^{-1} \in H$, so inverses are in $H$.

Also, $ab^{-1}(b^{-1})^{-1} \in H$, and
$ab^{-1}(b^{-1})^{-1}(b^{-1})^{-1} = ab \in H$, so $H$ is closed.

$\implies H\leq G$
$\quad \blacksquare$

\newpage
\paragraph{54} Show that the intersection $H\cup K$ of subgroups $H$ and $K$ of a
group $G$ is a subgroup of $G$.

\uwave{pf.}

$H\leq G$ and $K\leq G$ $\implies e\in H\cup K$. And that for any
three elements in both associativity holds as it is inherited from
$G$.

Furthermore, $a\in H$ and $a\in K$, implies $a^{-1} \in H$, and
$a^{-1}\in K$, so $a^{-1}$ is in the intersection.

$\blacksquare$

\paragraph{57} Show that a group with no proper non trivial subgroups
is cyclic.

\uwave{pf.}
Let $G$ be a group.

If $a,b\in G, a\neq e, b\neq e$, then $G$ is not cyclic if there is no
power of $a$ that you can raise it to get $b$. If $G$ is not cyclic
with that hypothesis, then $<b>$ must be a non trivial subgroup of
$G$, which is a contradiction$\quad \blacksquare$

\subsection*{Section 8}

\[\sigma = \begin{pmatrix}1&2&3&4&5&6\\3&1&4&5&6&2 \end{pmatrix},\quad \tau
  = \begin{pmatrix}1&2&3&4&5&6\\ 2&4&1&3&6&5 \end{pmatrix},\quad \mu
  = \begin{pmatrix}1&2&3&4&5&6\\ 5&2&4&3&1&6 \end{pmatrix}\]


\paragraph{2} $\tau^2\sigma$

\uwave{slu.}

$\tau^2 = \begin{pmatrix}1&2&3&4&5&6 \\ 4&3&2&1&5&6 \end{pmatrix}$

$\tau^2\sigma = \begin{pmatrix}1&2&3&4&5&6 \\
  2&4&1&5&6&3 \end{pmatrix}\quad \blacklozenge$

\paragraph{5} $\sigma^{-1}\tau\sigma$

This is just stacking the powers of $\sigma$.

$\begin{bmatrix}
  1&2&3&4&5&6\\
  3&1&4&5&6&2\\
  4&3&5&6&2&1\\
  5&4&6&2&1&3\\
  6&5&2&1&3&4\\
  2&6&1&3&4&5\\
  1&2&3&4&5&6
\end{bmatrix}$

So, $\sigma^6 = \iota \implies \sigma^{5} = \sigma^{-1}$.

$\sigma^{-1} = \begin{pmatrix} 1&2&3&4&5&6\\2&6&1&3&4&5\end{pmatrix}$


$\begin{bmatrix}
 \iota&                 1&2&3&4&5&6\\
 \sigma&                3&1&4&5&6&2\\
 \tau&                  2&4&1&3&6&5\\
 \tau\sigma&            1&2&3&6&5&4 \\
 \sigma^{-1}&           2&6&1&3&4&5\\
 \sigma^{-1}\tau\sigma& 2&6&1&5&4&3
\end{bmatrix}$

$\sigma^{-1} = \begin{pmatrix}
  1&2&3&4&5&6\\2&6&1&5&4&3\end{pmatrix}\quad \blacklozenge$

\newpage
\paragraph{6} $|\langle{\sigma}  \rangle|$

\uwave{slu.} $|\langle{\sigma}  \rangle| = 6$, because by the
computation above
it is the smallest power of $\sigma$ needed to get
$\iota\quad \blacklozenge$


\paragraph{7} $|\langle \tau^2 \rangle|$

\uwave{slu.}

$\begin{bmatrix}
 \iota&                 1&2&3&4&5&6\\
 \tau^2&                4&3&2&1&5&6\\
 (\tau^2)^2&            1&2&3&4&5&6
\end{bmatrix}$

So,$|\langle \tau^2 \rangle|=2$


\end{document}


%%% Local Variables:
%%% mode: latex
%%% TeX-master: t
%%% End:
