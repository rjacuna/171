\documentclass{article}
\usepackage{fontspec}
\usepackage{xcolor}
\usepackage{sagetex}

\usepackage{euler}
\usepackage{amsmath}
\usepackage{amssymb}
\usepackage{unicode-math}


\usepackage[makeroom]{cancel}
\usepackage{ulem}

\setlength\parindent{0em}
\setlength\parskip{0.618em}
\usepackage[a4paper,lmargin=1in,rmargin=1in,tmargin=1in,bmargin=1in]{geometry}

\setmainfont[Mapping=tex-text]{Helvetica Neue LT Std 45 Light}

\newcommand\N{\mathbb{N}}
\newcommand\Z{\mathbb{Z}}
\newcommand\R{\mathbb{R}}
\newcommand\C{\mathbb{C}}
\newcommand\A{\mathbb{A}}

\usepackage[thinlines]{easytable}
\begin{document}

\begin{center}
  171 --- 6

  RJ Acuna

  (862079740)
\end{center}\vspace{1.618em}


% \subsection*{Section 10}

% \paragraph{35*}

\subsection*{Section 11}

\paragraph{16} Are the groups $\Z_{2}\times\Z_{12}$ and
$\Z_4\times\Z_6$ isomorphic? Why or why not?

\uwave{slu.}
\begin{align*}
  \Z_{2}\times\Z_{12} &\simeq \Z_{2}\times
                        \Z_{4}\times\Z_{3} \text{ Since
                        gcd$(4,3) = 1$} \\
                      &\simeq \Z_{4}\times
                        \Z_2\times\Z_{3} \text{ By
                        rearrangement} \\
                      &\simeq \Z_4\times\Z_6\hspace{2.4em} \text{ Since
                        gcd $(2,3) = 1$}\quad \blacklozenge
\end{align*}

\paragraph{18} Are the groups $\Z_8\times \Z_{10} \times \Z_{24}$ and
$\Z_4\times \Z_{12} \times \Z_{40}$ isomorphic? Why or why not?

\uwave{slu.}
\begin{align*}
\Z_8\times \Z_{10} \times \Z_{24} &\simeq \Z_8\times \Z_2\times \Z_{5}
                                    \times \Z_{8}\times\Z_{3}\quad \text{
                                    Since gcd$(2,5)=1$ and gcd$(8,3)=1$}\\
 &\simeq \Z_8\times \Z_2\times \Z_{40}\times\Z_{3} \text{
   Since gcd$(8,5)=1$}\\
   &\simeq \Z_8\times \Z_2 \times\Z_{3}\times \Z_{40} \text{ By
     rearrangement}\\
                                  &\text{and}\\
\Z_4\times \Z_{12} \times \Z_{40} &\simeq \Z_4\times \Z_4\times \Z_{3}
                                    \times \Z_{40}\quad \text{
                                    Since gcd$(4,3)=1$}\\
  &\text{but}\\
\text{lcm}(8,2)=8 \text{ and } \text{lcm}(4,4)=4 &\implies |\Z_8\times
\Z_2|=8 \text{ and } |\Z_4\times \Z_4|=4 \\ &\implies
                                              \Z_8\times \Z_2 \not\simeq \Z_4\times\Z_4
\end{align*}

So, no $\blacklozenge$

\paragraph{24} Are the groups $\Z_4\times \Z_{18} \times \Z_{15}$ and
$\Z_3\times \Z_{36}\times \Z_{10}$ isomorphic? Why or why not?

\uwave{slu.}
\begin{align*}
  4=2^2,18=2\cdot 3^2, 15 = 3\cdot 5&\text{ and }3 = 3, 36 = 2^2\cdot3^2,10
                                      = 2\cdot 5 \\ &\implies \\
  \Z_4\times \Z_{18} \times \Z_{15} &\simeq  \Z_{2^2}\times \Z_{2}
                                      \times \Z_{3^2} \times \Z_{3}\times \Z_{5}\\
                                      &\simeq  \Z_{2^2}\times \Z_{3}
                                        \times \Z_{3^2} \times \Z_{2}\times \Z_{5} \\
  &\simeq  \Z_{3}\times \Z_{2^2}
    \times \Z_{3^2} \times \Z_{2}\times \Z_{5} \\
                                    &\simeq \Z_3\times \Z_{36}\times \Z_{10}\\
\end{align*}

So, yes $\blacklozenge$

% \paragraph{29*}
% \paragraph{39*}
% \paragraph{40*}
% \paragraph{41*}
% \paragraph{42*}

% \paragraph{46**}
\newpage
\paragraph{49} Find a counterexample of Exercise $47$ with the
hypotheses of that $G$ is abelian omitted.

\uwave{slu.}

\begin{sagesilent}
  from sage.combinat.permutation import *
  Permutations.options.latex='cycle'
  S_3 = [Permutation(x)  for x in Permutations(3).list()]
  S_3_latex = [latex(x) for x in S_3]
  squared_latex = ["%s^2 = %s\n"%(latex(x),latex(x.right_action_product(x)))
                    for x in S_3]
  squared_latex2 = [x.replace("()","\\iota") for x in squared_latex[1:]]
  squares = [x for x in S_3 if x.right_action_product(x) == S_3[0]]
  squares_latex = ["\\enskip%s"%(latex(x)) for x in squares]
  a = squares[1].left_action_product(squares[2])
\end{sagesilent}

$$S_3 = \left\{\iota,\sagestr{", ".join(S_3_latex[1:])}\right\}$$
If you square each element you get
$$\sagestr{", ".join(squared_latex2)}.$$
Then the set of squares and $\iota$ is, $S = \left\{\iota, \sagestr{", ".join(squares_latex[1:])}\right\}$.

$\sage{squares[1]}\sage{squares[2]} = \sage{a}\not\in S
\implies S$ is not closed $\implies S$ is
not a subgroup of $S_4 \quad \triangleright\triangleleft$

I did the previous by brute force with the computer.

However, I found the smallest example up to isomorphism, consider $D_3$:

\begin{TAB}(e,0.5cm,0.5cm){|c|c:c:c:c:c:c|}{|c|c:c:c:c:c:c|}
  $D_3$&1&A&B&C&D&E\\
  1&1&A&B&C&D&E\\
  A&A&B&1&D&E&C\\
  B&B&1&A&E&C&D\\
  C&C&E&D&1&B&A\\
  D&D&C&E&A&1&B\\
  E&E&D&C&B&A&1
\end{TAB}

$\implies S = \{1,C,D,E\} \land CD=B \not\in S \implies
S\not\leq D_3 \triangleright\triangleleft$

Ok, this is annoying. $D_3 \simeq S_3$... Looking at the table
I've realized they are the same thing. I wanted to consider a smaller
example so I googled what is the smallest non-abelian group. So, I had
already found the smallest example.

% \paragraph{50**}

\subsection*{Section 13}

In Exercises 1 through 15, determine whether the given map $\phi$ is
a homomorphism. [\textit{Hint:} the straightforward way to proceed is
to check whether $\phi(ab) = \phi(a)\phi(b)$ for all $a$ and $b$ in
the domain of $\phi$. However, if we should happen to notice that
$\phi^{-1}[\{e^\prime\}]$ is not a subgroup whose left and right
cosets coincide, or that $\phi$ does not satisfy the properties given
in Exercises $44$ or $45$ for finite groups, then we can say at once
that $\phi$ is not a homomorphism.]

% \paragraph{1**}

\paragraph{2} Let $\phi:\R\rightarrow \Z$ under addition be given by
$\phi(x) =$ the greatest integer $\leq x.$

\uwave{slu.} No,  $\phi(1.5 + 1.5) = \phi(3) = 3 \neq 2 = 1 + 1 =
\phi(1.5) + \phi(1.5) \quad \blacklozenge$

% \paragraph{3**}
% \paragraph{4**}
% \paragraph{5**}
% \paragraph{6**}
% \paragraph{7**}
% \paragraph{8**}
% \paragraph{9**}
% \paragraph{10**}
% \paragraph{11**}

\paragraph{12} Let $M_n$ be the additive group of all $n\times n$
matrices with real entries, and let $\R$ be the additive group of real
numbers. Let $\phi(A) = \det(A)$

\uwave{slu.} No, $\det\left(\begin{pmatrix}1&0\\
  0&0\end{pmatrix}+\begin{pmatrix}0&0\\ 0&1\end{pmatrix}\right)=
\det\left(\begin{pmatrix}1&0\\ 0&1\end{pmatrix}\right) = 1\neq 0 = 0+0 = \det\left(\begin{pmatrix}1&0\\ 0&0\end{pmatrix}\right)+\det\left(\begin{pmatrix}0&0\\ 0&1\end{pmatrix}\right) \quad \blacklozenge$

\newpage
\paragraph{13} Let $M_n$ and $R$ be as in Exercise $12$. Let
$\phi(A)=\text{tr}(A)$ for $A\in M_n$, where the \textbf{trace} of $A$
is the sum of the elements on the main diagonal of $A$ from the upper-left to the lower-right corner.

\uwave{slu.} Yes,
\begin{align*}
  \text{tr}(A+B) &= \text{tr}((A_{ij})+(B_{ij}))\\
                 &= \text{tr}((A_{ij}+B_{ij})) \\
                 &= \sum_{i=1}^n A_{ii} + B_{ii}\\
                 &= \sum_{i=1}^n A_{ii} + \sum_{i=1}^n B_{ii}\\
                 &= \text{tr}((A_{ij})) + \text{tr}((B_{ij})) \\
                 &= \text{tr}(A) + \text{tr}(B) \quad \blacklozenge
\end{align*}

\paragraph{14} Let $GL(n,\R)$ be the multiplicative group of
invertible $n\times n$ matrices, and let $\R$ be the additive group of
real numbers. Let $\phi: GL(n,\R)\rightarrow \R$ be given by $\phi(A)
= \text{tr}(A),$ where $\text{tr}(A)$ is defined as in Exercise $13.$

\uwave{slu.} No,
\begin{align*}
  \text{tr}\left(\begin{pmatrix}1&0\\
  0&0\end{pmatrix}\begin{pmatrix}0&0\\ 0&1\end{pmatrix}\right) = \text{tr}\left(\begin{pmatrix}0&0\\
  0&0\end{pmatrix}\right)
                                 = 0 \neq 2
                                 = 1 + 1
                                 = \text{tr}\left(\begin{pmatrix}1&0\\
  0&0\end{pmatrix}\right) + \text{tr}\left(\begin{pmatrix}0&0\\
  0&1\end{pmatrix}\right)\quad \blacklozenge
\end{align*}
% \paragraph{15**} ...


In Exercises $16$ through $24$, compute the indicated quantities for
the given homomorphism $\phi$. (See Exercise $46.$)

% \paragraph{16**}
% \paragraph{17**}
% \paragraph{18**}
% \paragraph{19**}
% \paragraph{20**}

\paragraph{21} Ker$(\phi)$ and $\phi(14)$ for $\phi: \Z_{24}\rightarrow
S_8$ where $\phi(1) = (2, 5)(1, 4, 6, 7)$

\begin{sagesilent}
  from sage.combinat.permutation import *
  Permutations.options.latex='cycle'
  s0 = Permutation('(2, 5)(1, 4, 6, 7)')
  def pow_perm(x,n):
      b = x
      c = 1
      while (c < n):
          b = x.right_action_product(b)
          c += 1
      return b

  sps = ["\\phi(%s) = [%s]^%s = %s"%(n,latex(s0),n,latex(pow_perm(s0,n)).replace("()","\\iota")) for n in range(1,5)]
  s14 = pow_perm(s0,14)
\end{sagesilent}

\uwave{slu.}
$$\sagestr{", ".join(sps[1:])}.$$
We can see from the computation that, multiples of $4$ give us the
identity.
So, Ker$(\phi) = \{0,4,8,12,16,20\}$
\begin{align*}
\phi(14) &= [\sage{s0}]^{14} = [\sage{s0}]^{12+2}\\
 &=([\sage{s0}]^4)^3 [\sage{s0}]^2 =
   \iota^3[\sage{s0}]^2 = \sage{s14}\quad \blacklozenge
\end{align*}

% \paragraph{22**}
% \paragraph{23**}
% \paragraph{24**}
% \paragraph{25**}
% \paragraph{26**}
% \paragraph{27**}

In Exercises $33$ through $43$, give an example of a nontrivial
homomorphism $\phi$ for the given groups, if an example exists. If no
such homomorphism exists explain why that is so. You may use
Excercises $44$ and $45$.

% \paragraph{33**}
% \paragraph{34**}
% \paragraph{35**}
% \paragraph{36**}
% \paragraph{37**}
% \paragraph{38**}
% \paragraph{39**}
% \paragraph{40**}
\newpage
\paragraph{41} $\phi: D_4\rightarrow S_3$

\uwave{slu.}

From problem $49$ we can see $S_3\simeq D_3$
\begin{align*}
  D_n &= \langle r,s|r^n=s^2 = e, srs = r^{-1}\rangle\\
  &\implies\\
  D_4 &= \langle r,s|r^4=s^2 = e, srs = r^{-1}\rangle\\
  &\text{and}\\
  D_3 &= \langle a,b|a^3=b^2 = e', bab = a^{-1}\rangle\\
\end{align*}
So, let's check,
$$\nu(r)=a \implies \nu(e) = \nu(r^4) = \nu(r)^4 = \nu(r)^3\nu(r)
= a^3\phi(r) =e'^3a = a \neq e'$$.
So, $\nu(r)\neq a$. Check,
$$\nu(r)=b \implies \nu(e) = \nu(r^4) = \nu(r)^4 = b^4
= (b^2)^2 =e'^2 = e'$$.
So, $ \nu(s) b \nu(s) = \nu(s)\nu(r)\nu(s)=\nu(srs)=\nu(r^{-1}) =\nu(r)^{-1} =
b^{-1} = b$. So, $\nu(s) = e'$.

Since, all the elements of $D_4$ are, $e, r,r^2,r^3,
s,sr,sr^2,sr^3$. Then define $\psi$ by,

$\psi(e) = e'$,

$\psi(r) = b$,

$\psi(r^2) = b^2 = e'$,

$\psi(r^3) = be' = b$,

$\psi(s) = e'$,

$\psi(sr) = e'b = b$,

$\psi(sr^2) = e'b^2 = e'$,

$\psi(sr^3) = e'b = b$,

Now construct an isomorphism $\mu$ from $D_3$ to $S_3$ like so,

$\mu(e)  = \iota$,

$\mu(a)  = (1 2 3)$,

$\mu(a^2)  = (1 3 2)$

$\mu(b)  = (2 3)$

$\mu(ba)  = (1 2)$

$\mu(ba^2)  = (1 3)$.

$\mu$ is a bijection from $D_3$ to $S_3$. It satisfies the
homomorphism property by construction, because I just looked at the
tables of $D_3$ and $S_3$ to write it.

Now, put $\phi = \psi\circ\mu$
which gives,

$\phi(r) = \phi(r^3) = \phi(sr) = \phi(sr^3) = (2 3)$,
$\phi(e) = \phi(r^2) = \phi(s) = \phi(sr^2) =e'$

Which is nontrivial $\blacklozenge$

% \paragraph{42**}
\newpage
\paragraph{47} Show that any group homomorphism $\phi: G\rightarrow
G^\prime$ where $|G|$ is a prime must either be the trivial
homomorphism or a one-to-one map.

\uwave{pf.}

$|G|$ is prime $\implies$ $G$ has no proper nontrivial subgroups.

Ker$(\phi) \leq G \implies$ Ker$(\phi) = \{e\}\lor$ Ker$(\phi) = G$

Ker$(\phi) = \{e\} \implies \phi$ is injective.

Ker$(\phi) = G \implies \phi$ is trivial $\quad \blacksquare$

\paragraph{49} Show that if $G$, $G'$, and $G^{''}$ are groups and if
$\phi: G\rightarrow G'$ and $\gamma:G'\rightarrow G^{''}$ are
homomorphisms, then the composite map $\gamma\phi:G \rightarrow
G^{''}$ is a homomorphism.

\uwave{pf.}

WTS $\forall x,y \in G: \gamma\phi(xy) = \gamma\phi(x)\gamma\phi(y)$

Let $x,y \in G$ be arbitrary.

Compute,
\begin{align*}
  \gamma\phi(xy) &= \gamma(\phi(xy))\\
                 &= \gamma(\phi(x)\phi(y)) \\
                 &= \gamma(\phi(x))\gamma(\phi(y))\\
                 &= \gamma\phi(x)\gamma\phi(y)
\end{align*}
% \paragraph{55*}

\subsection*{Additional Exercises}

\paragraph{1} Let $\phi: G\rightarrow G^\prime$ be a homomorphism of
groups and suppose $g\in G$ is an element of finite order.

(a) Prove that $|\phi(g)|$ divides $|g|$.

\uwave{pf.}

Let $|g| = n$, $g^n = e\implies \phi(g)^n = \phi(g^n) = g(e) = e'$.

Now,$\phi(g)^n = e'$ doesn't mean $|\phi(g)| = n$.

But, it does mean that $\exists k,m\in\Z^+: n = mk \land g^m = e'$, where $m$ is the
order of $\phi(g)$.

So, the order of $\phi(g)$ divides the
order of $g\quad \blacksquare$.

(b) Prove that $|\phi(g)| = |g|$ if $\phi$ is injective.

\uwave{pf.}

$\phi(g)^n = \phi(g^n) = \phi(e) = e'$.

Since, $\phi$ is injective no power smaller than $n$ maps $g^n$ to
$e'$ under $\phi$.

So, the smallest power you need to raise $\phi(g)$ to get $e'$ is $n.$

So, $\phi$ is a bijection. So $|\phi(g)| = |g|\quad \blacksquare$


\paragraph{2**} Let $G$ and $G^\prime$ be groups. Prove that a
homomorphism $\phi: G\rightarrow G^\prime$ is an isomorphism if and
only if there exists some homomorphisms $\psi:G^\prime\rightarrow G$
such that $\psi\circ \phi = \text{id}_G$ and $\phi\circ \psi = \text{id}_{G^\prime}$.

\uwave{pf.}

$(\implies)$ Ass. $\phi:\langle
G,\bullet\rangle\rightarrow \langle G',* \rangle$ is an isomorphism

Then $\phi$ is bijective, and

$$\forall x,y \in G: \phi(x\bullet y) =\phi(x)*\phi(y)$$

$\phi$ is bijective, so it has an inverse mapping $\phi^{-1}$.

That is $\phi\circ \phi^{-1} = 1_G$ and $\phi^{-1}\circ \phi = 1_G'$

So, for all $x,y\in G$ we can write $x = \phi^{-1}(\phi(x))$ and $y =
\phi^{-1}(\phi(y))$. Let $\phi(x) = x'$ and  $\phi(y) = y'$.
\begin{align*}
  \phi^{-1}(\phi(x\bullet y)) &=\phi^{-1}(\phi(x)*\phi(y))\\
  x\bullet y &= \phi^{-1}(\phi(x)*\phi(y))\\
  \phi^{-1}(\phi(x))\bullet \phi^{-1}(\phi(y)) &=
                                                 \phi^{-1}(\phi(x)*\phi(y))\\
  \phi^{-1}(x')\bullet \phi^{-1}(y') &= \phi^{-1}(x'*y')
\end{align*}

So, $\phi^{-1}$ satisfies the homomorphism property, thus is a
homomorphism as in the theorem.

$(\impliedby)$ Ass. $\exists$ homomorphisms $\phi,\psi: \phi\circ\psi = 1_{G'}$ and
$\psi\circ\phi = 1_G$

$\psi\circ\phi = 1_G \implies \psi = \phi^{-1} \implies \phi$ is
bijective.

$\phi$ is a homomorphism, so $\phi$ is an isomorphism.

Thus, proven. This holds for binary algebraic structures, so it holds
for groups.

$\blacksquare$

\paragraph{3} Show that $SL_n(\R)$ is a normal subgroup of $GL_n(\R)$
for all $n \in \Z^+$.

\uwave{pf.}

The determinant det:$GL_n(\R) \rightarrow \R^*$ is a
homomorphism.

$\det^{-1}[\{1\}] = SL_n(\R)$ by definition of $SL_n(\R)$.
So, Ker(det)=$SL_n(\R)$.

By $13.20$ $SL_n(\R)$ is normal for each $n \in \Z^+ \quad \blacksquare$


\end{document}


%%% Local Variables:
%%% mode: latex
%%% TeX-master: t
%%% End:
