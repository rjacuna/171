\documentclass{article}
\usepackage{fontspec}
\usepackage{xcolor}
\usepackage{sagetex}

\usepackage{euler}
\usepackage{amsmath}
\usepackage{amssymb}
\usepackage{unicode-math}


\usepackage[makeroom]{cancel}
\usepackage{ulem}

\setlength\parindent{0em}
\setlength\parskip{0.618em}
\usepackage[a4paper,lmargin=1in,rmargin=1in,tmargin=1in,bmargin=1in]{geometry}

\setmainfont[Mapping=tex-text]{Helvetica Neue LT Std 45 Light}

\newcommand\N{\mathbb{N}}
\newcommand\Z{\mathbb{Z}}
\newcommand\R{\mathbb{R}}
\newcommand\C{\mathbb{C}}
\newcommand\A{\mathbb{A}}

\usepackage[thinlines]{easytable}
\begin{document}

\begin{center}
  171 --- 8

  RJ Acuna

  (862079740)
\end{center}\vspace{1.618em}

$\forall G, G': [G\times G',G\times G'] = [G,G]\times[G',G']$

\uwave{pf. }

Since, $\forall$ groups $Z$, $[Z,Z]$  is subgroup, so in particular it
is a group. And the direct product of groups is a group by the
operation taken componentwise. It follows that, $\forall$ groups
$X,Y,$ $[X\times Y, X\times Y]$ and $[X,X]\times [Y,Y]$ have compatible
group structures. Therefore, it is enough to show containment both
ways.

Let $(x_1,y_1)(x_2,y_2)(x_1,y_1)^{-1}(x_2,y_2)^{-1}\in$ $[X\times Y,
X\times Y]$, and $(x_1x_2x_1^{-1}x_2^{-1},y_1y_2y_1^{-1}y_2^{-1}) \in
[X,X]\times [Y,Y]$.
\begin{align*}
[X\times Y,
X\times Y] \ni (x_1,y_1)(x_2,y_2)(x_1,y_1)^{-1}(x_2,y_2)^{-1} &=
                                                 (x_1,y_1)(x_2,y_2)(x_1^{-1},y_1^{-1})(x_2^{-1},y_2^{-1})\\
                                               &=
                                                 (x_1x_2x_1^{-1}x_2^{-1},y_1y_2y_1^{-1}y_2^{-1})
  \in [X,X]\times [Y,Y]\\
\end{align*}
Since they are both arbitrary elements the conclusion follows by
letting $X = G$, and $Y = G'\quad \blacksquare$
\subsection*{Section 15}

\paragraph{13} Find both the center $Z(D_4)$ and the commutator
subgroup $C$ of the group $D_4$ of symmetries of the square in table
$8.12.$

\uwave{pf.}

Going along the slant diagonals of the table we can eliminate the
diagonals that don't have symmetry about the main diagonal. Therefore,
the only elements that commute with everything else are $\rho_0$ and
$\rho_2$. So, $Z(D_4) =\{\rho_0,\rho_2\}.$

We know that if $D_4/N$ is abelian then $[D_4,D_4]\leq N$.

\paragraph{14}

\paragraph{15}

\paragraph{30}

\paragraph{31}

\paragraph{37}

\subsection*{Section 35}

\paragraph{13}

\paragraph{14}

\paragraph{18}

\paragraph{19}

\subsection*{Additional Exercise(s)}

\paragraph{1} Prove that every nilpotent group is solvable.

\paragraph{2} Determine the composition factors of $\Z_{48}$ and
$\Z_{60}$ (with multiplicity).






\end{document}


%%% Local Variables:
%%% mode: latex
%%% TeX-master: t
%%% End:
