\documentclass{article}
\usepackage{fontspec}
\usepackage{xcolor}
%\usepackage{sagetex}

\usepackage{euler}
\usepackage{amsmath}
\usepackage{unicode-math}


\usepackage[makeroom]{cancel}
\usepackage{ulem}

\setlength\parindent{0em}
\setlength\parskip{0.618em}
\usepackage[a4paper,lmargin=1in,rmargin=1in,tmargin=1in,bmargin=1in]{geometry}

\setmainfont[Mapping=tex-text]{Helvetica Neue LT Std 45 Light}

\newcommand\N{\mathbb{N}}
\newcommand\Z{\mathbb{Z}}
\newcommand\R{\mathbb{R}}
\newcommand\C{\mathbb{C}}
\newcommand\A{\mathbb{A}}

\usepackage{soul}
\begin{document}

\begin{center}
  Chapter 1 --- Affine Algebraic Sets
\end{center}\vspace{1.618em}

My construction of the natural numbers follows Halmos. However, I
claim $0 \notin \N$, minor modifications to the theory are required
and are sketched as follows. I leave the details to the interested
reader. I worked them out but my computer died so I lost the .tex
files, now they only exist in .pdf form in an email I sent to one of
my friends.

Put, $\N := \{1, 2, 3, 4, 5, \dots\}$ the natural numbers. Define, $|\emptyset| :=
0$. To count is to give an isomorphism between a set and a standard subset of the
natural numbers. An isomorphism is a one to one and onto map that
preserves the algebraic structure of the domain and the range. At the
level of sets there is none, however we still like to talk about
isomorphism to keep the language consistent.

Standard subsets of $\N$ are sequences
containing $1$, increasing by $1$ and ending in some element of
$\N$. We call sets that cannot be put into one to one correspondence
with a standard subset infinite. If we can put a set into one to one
correspondence with $\N$ we say it is countable. The concept of
cardinality is defined this way for finite and infinite sets. The
cardinality of $\N$, is called $|\N|:= \aleph_0$ aleph null, or aleph
naught. If a set is neither finite, nor countable, we say it is
uncountable and it has higher cardinality.

We define an order on $\N$, by the natural chain of standard sets,

\[\{1\} \subset \{1,2\}\subset \{1,2,3\} \subset \dots \subset \N \]

That is $\{1\} \subset \{1,2\}$, so $|\{1\}| < |\{1,2\}|$, and so
on. In a weaker sense $\leq$ is defined by not requiring the left hand
side to be a proper subset.

The Cartesian product has two interpretations,

\[\prod_{i \in I} X_i = \{(x_i)_{i \in I} | x_i\in X_i\} =
  \{f:I \rightarrow \bigcup_{i\in I} X_i | (\forall i\in I)(f(i) \in X_i)\}\]


First, it is the set of all tuples, where each tuple has size $|I|$, where
$x_j$ is the $j^{th}$ coordinate of $(x_i)_{i\in I}$.

Second, it is also the set of all maps $f$ such that $i\mapsto f(i)$. From the
second interpretation we can recover the concept of coordinates by
defining the projection maps,

\[\pi_j : \prod_{i\in I} X_i \rightarrow X_j; f\mapsto f(j)\]

Then, $\pi_k(f) = f(k)$ is the component of $f$ in the $k$ direction.

When $X_i = X$ for all $i\in I$, and $I\subset \N$, with $n = |I|$, we
call this the Cartesian power, and
we write $X^n$.

An operation on a set $S$ is a map $*:S^n\rightarrow S$. The size $n$
of the input vector is the arity of the operation. We give special
names to the first few naturals 1, 2, and 3-arity operations are
called unary, binary, and ternary operations respectively. For
operations with arity $n$ we say they are $n$-ary.

Now, 0 arity operations are special, they are called nullary operations. First we need to give meaning to $S^0$. To
accomplish this we set $I = \emptyset$. We define,

\[X^0 := \prod_{i \in \emptyset} X = \{(x_i)_{i \in \emptyset} | x_i\in X\} =
  \{f:\emptyset \rightarrow \bigcup_{i\in \emptyset} X_i | (\forall
  i\in \emptyset)(f(i) \in X_i)\}\]

In the first interpretation since there is nothing in the empty set,
it follows
$X^0 = \{()\}$, where $()$ has no coordinates, it is called the empty
tuple.

In the second interpretation, $\forall i\in \emptyset$ is unbound as it determines no values $i$. Furthermore,
$\bigcup_{i\in \emptyset} X_i = \emptyset$ for the same reason. So,
$X^0 = \{f:\emptyset\rightarrow \emptyset \}$.

Recall from Na\"ive Set Theory (Halmos), that a function is a subset of the Cartesian
product of its domain and range. $\emptyset \times \emptyset =
\emptyset \implies X^0 = \{\emptyset \}$.

In both cases $|X^0| = 1$. So, any nullary operation $S^0\rightarrow
S$ is just an element of $S$. The set of all nullary operations on
$S$, is in one to one correspondence with the elements of $S$.

If $|I| \geq \aleph_0$, then we need to accept the Axiom of Choice in
order to ensure that the Cartesian product is not empty.

\end{document}

%%% Local Variables:
%%% mode: latex
%%% TeX-master: t
%%% End:
