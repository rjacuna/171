\documentclass{article}
\usepackage{fontspec}
\usepackage{xcolor}
%\usepackage{sagetex}

\usepackage{euler}
\usepackage{amsmath}
\usepackage{amssymb}
\usepackage{unicode-math}


\usepackage[makeroom]{cancel}
\usepackage{ulem}

\setlength\parindent{0em}
\setlength\parskip{0.618em}
\usepackage[a4paper,lmargin=1in,rmargin=1in,tmargin=1in,bmargin=1in]{geometry}

\setmainfont[Mapping=tex-text]{Helvetica Neue LT Std 45 Light}

\newcommand\N{\mathbb{N}}
\newcommand\Z{\mathbb{Z}}
\newcommand\R{\mathbb{R}}
\newcommand\C{\mathbb{C}}
\newcommand\A{\mathbb{A}}

\usepackage[thinlines]{easytable}
\begin{document}

\begin{center}
  171 --- 7

  RJ Acuna

  (862079740)
\end{center}\vspace{1.618em}


\subsection*{Section 14}

In Exercises $1$ through $8$, find the order of the given factor
group.

\paragraph{2} $(\Z_4\times \Z_{12})/(\langle 2 \rangle\times \langle 2  \rangle)$

\uwave{slu.}

$2\cdot 2 = 4 \equiv 0 \text{ (mod $4$)} \implies |2| = 2$ in $\Z_4$

$2\cdot 6 = 12 \equiv 0 \text{ (mod $12$)} \implies |2| = 6$ in $\Z_4$

$\implies |\langle 2 \rangle\times \langle 2  \rangle| = 2\cdot 6 =
12 \quad \lozenge$

$|\langle 2 \rangle\times \langle 2  \rangle|
12 \land |\Z_4\times \Z_{12}| = 4\cdot 12 \implies |(\Z_4\times
\Z_{12})/(\langle 2 \rangle\times \langle 2  \rangle)| = 4$

\paragraph{6} $(\Z_{12}\times \Z_{18})/\langle (4,3) \rangle$

$4\cdot 3 = 12 \equiv 0 \text{ (mod $12$)} \implies |4| = 3$ in $\Z_{12}$

$3\cdot 6 = 18 \equiv 0 \text{ (mod $12$)} \implies |3| = 6$ in $\Z_{18}$

$\text{lcm}(3,6) = 6 \implies |\langle (4,3) \rangle| =  6$

$|\langle (4,3) \rangle| = 6 \land |\Z_{12}\times \Z_{18}| = 12\cdot
18 = 2\cdot 6\cdot 18 \implies |(\Z_{12}\times \Z_{18})/\langle (4,3)
\rangle| = 2\cdot 18 = 36 \quad \lozenge$

\vspace{1em}

In Exercises $9$ through $15$, give the order of the element in the
factor group.
\paragraph{10} $26 + \langle 12 \rangle$ in $\Z_{60}/\langle 12
\rangle$

\uwave{slu.}

$26 = 12\cdot 2 + 2 \implies 26 + \langle 12 \rangle = 2 + \langle 12
\rangle$

$2\cdot 6 = 12 \implies (2+\langle 12 \rangle)^6 = 2\cdot 6 +\langle
12 \rangle  = 12 +\langle 12 \rangle = \langle 12 \rangle \implies
|26+\langle 12 \rangle| = 6$ in $\Z_{60}/\langle 12
\rangle\quad \lozenge$

\paragraph{14} $(3,3)+\langle (1,1) \rangle$ in $(\Z_4\times \Z_4)/\langle (1,1) \rangle$

\uwave{slu.}

$(3,3) = 3(1,1) \implies (3,3)\in \langle (1,1) \rangle \implies
(3,3)+\langle (1,1) \rangle = \langle (1,1) \rangle $

$\implies |(3,3) +
\langle (1,1) \rangle| = 1$ in $(\Z_4\times \Z_4)/\langle (1,1)
\rangle \quad \lozenge$

\paragraph{16} Compute $i_{\rho_1}[H]$ for the subgroup $H=
\{\rho_0,\mu_1\}$ of the group $S_3$ of Example $8.7.$

\uwave{slu.}

$\rho_1\rho_0\rho_1^{-1}=\rho_1\rho_1^{-1} = \rho_0$

and

$\rho_1\mu_1\rho_1^{-1} = \rho_1\mu_1\rho_2 = \mu_3\rho_2 = \mu_2$

so, $i_{\rho_1}[H] = \{\rho_0,\mu_2\} \quad \lozenge$

\paragraph{24} Show $A_n$ is a normal subgroup of $S_n$ and compute
$S_n/A_n;$ that is find a known group to which $S_n/A_n$ is
isomorphic.

\uwave{pf. } $\quad |S_n| = n!\land |A_n| = \frac{n!}{2} \implies
|S_n/A_n| = \frac{n!}{\frac{n!}{2}} = 2 \implies S_n/A_n \simeq \Z_2$
$\quad \blacksquare$

\paragraph{26} Prove that the torsion subgroup $T$ of an abelian group
$G$ is a normal subgroup of $G$, and that $G/T$ is torsion free.

\uwave{pf. }

Note that $T\trianglelefteq G \iff \forall g\in G,t \in T, gtg^{-1}
\in T.$

$G$ is abelian, and $T\leq G$. Let $g\in G,$and $t\in T$, then
$gtg^{-1} = gg^{-1}t = et = t\in T\implies T\trianglelefteq G$

Suppose for the sake of contradiction $\exists g \in G,g\not\in T,
n\in\Z: g\neq e \land (gT)^n = T$.

$ (gT)^n = g^n T = T \implies g^n = e \implies |g| = n$

Which is a contradiction, because $T$ contains all of the elements of
$G$ that have finite order.

So, the only element of $G/T$ of finite order is the coset $T$.

Therefore, $G/T$ is torsion-free $\quad \blacksquare$



\paragraph{30} Let $H$ be a normal subgroup of a group $G$, and let $m
= (G:H)$. Show that $a^m\in H$ for every $a\in G.$

\uwave{pf.} $\quad H\trianglelefteq G \land m = (G:H)=|G/H|\in \Z \implies \forall aH\in G/H,\quad (aH)^m = a^m H
= H \implies a^m \in H \quad \blacksquare$

\paragraph{34} Show that if a finite group $G$ has exactly one
subgroup $H$ of a given order, then $H$ is a normal subgroup of $G$.

\uwave{pf.}

WTS $\quad|G| = m \in \Z \land (\exists! H\leq G \land n\in\Z): |H| = n
\implies H\trianglelefteq G.$

WTS $\quad \forall g\in G, \forall h\in H,\enskip ghg^{-1} \in H \iff
H\trianglelefteq G$

Let $g\in G$, and $h\in H$, then $(ghg^{-1})^n =
ghg^{-1}ghg^{-1}\cdots ghg^{-1} = ghehehe\cdots hehg^{-1} = gh^ng^{-1}$.

$|H|=n \implies h^n = e \implies (ghg^{-1})^n = e \implies ghg^{-1} \in H$ since $\exists!
H\leq G: |H| = n\quad \blacksquare$

\subsection*{Section 15}

In Exercises $1$ through $12$, classify the given group according to
the fundamental theorem of finitely generated abelian groups.

4. $(\Z_4\times \Z_8)/\langle (1,2) \rangle$

$1\cdot 4 = 4 \equiv 0 \text{ (mod $4$)}\implies |1| = 4$ in $\Z_4$

and
$2\cdot 4 = 8 \equiv 0 \text{ (mod $8$)}\implies |2| = 4$ in $\Z_8$

lcm$(4,4) = 4 \implies |\langle (1,2) \rangle| = 4$ in $\Z_4\times
\Z_8$

$|\langle (1,2) \rangle| = 4$  and $|\Z_4\times \Z_8| = 4\cdot 8
\implies |(\Z_4\times \Z_8)/\langle (1,2) \rangle| = 8= 2^3$

So, there are three possibilities $\Z_8$, $\Z_4\times \Z_2$, $\Z_2\times\Z_2\times\Z_2$.

$ \langle (1,2) \rangle = \{(1,2),(2,4),(3,6),(0,0)\}$

$4(1,0) = (0,0) \in \langle (1,2) \rangle \implies |(1,0)+\langle
(1,2) \rangle| = 4$

$8(0,1) = (0,0) \in \langle (1,2) \rangle \implies  |(0,1) + \langle
(1,2) \rangle| = 8$

Since $\Z_8$ is the only choice that has an element of order $8$, it
follows that,$(\Z_4\times \Z_8)/\langle (1,2) \rangle \simeq \Z_8$
$\quad \blacklozenge$

\paragraph{8} $(\Z\times\Z\times\Z)/\langle (1,1,1)  \rangle$

\uwave{slu.}

$|(1,1,1)|=\aleph_0$ so we can't do simple counting arguments.

However, even though we don't have the notion of dimension.
We can see that $\langle (1,1,1) \rangle$ is the one dimensional span of the vector $(1,1,1)$
in the $\Z^3$ vector space. Therefore, a reasonable assumption is that
$(\Z\times\Z\times\Z)/\langle (1,1,1)  \rangle \simeq \Z^2$.

$(\Z\times\Z\times \Z) = \langle (1,0,0),(1,1,0),(1,1,1) \rangle$
\begin{align*}
  \forall w = (x,y,z)\in \Z^3: (x, y, z) &= (x-z,y-z,0)+ z(1,1,1)\\
                                         &= (x-y)(1,0,0) +
                                           (y-z)(1,1,0) +z(1,1,1)\\
                                         &= (x-y+y-z+z,y-z+z,z) = w
\end{align*}

$(\Z\times\Z\times \Z)/\langle (1,1,1) \rangle = \langle
(1,0,0),(1,1,0),(1,1,1) \rangle/\langle (1,1,1) \rangle = \langle
(1,0,0),(1,1,0) \rangle\simeq \Z^2 \quad \blacksquare$

\paragraph{10} $(\Z\times \Z\times \Z_8)/\langle (0,4,0) \rangle$

\uwave{slu.}

$\langle (0,4,0) \rangle \simeq 4\Z \implies (\Z\times \Z\times
\Z_8)/\langle (0,4,0) \rangle\simeq \Z\times \Z_4\times \Z_8 \quad
\blacksquare$

\paragraph{35} Let $\phi: G\rightarrow G'$ be a group homomorphism,
and let $N$ be a normal subgroup of $G$. Show that $\phi[N]$ is normal
subgroup of $\phi[G]$.

\uwave{pf.}

$\forall g\in G, n\in\N: gng^{-1} \in N \implies \phi(gng^{-1}) =
\phi(g)\phi(n)\phi(g^{-1}) = \phi(g)\phi(n)\phi(g)^{-1} \in \phi[N].$

But, $\forall g\in G, \phi(g)\in \phi[G]\land \forall n \in N,
\phi(n)\in N \implies
\phi[N]\trianglelefteq \phi[G]\quad \blacksquare$

\subsection*{Additional Exercises}

\paragraph{1} You showed on the last homework assignment that, for any $n ∈ \Z^+,SL_n(\R)$ is a normal
subgroup of $GL_n(\R)$. To what group that we’ve seen in this class is $GL_n(\R)/SL_n(\R)$
isomorphic? Prove your claim. [Note: Your answer shouldn’t depend on
$n$.]

\uwave{pf.}

$SL_n(\R)\trianglelefteq GL_n(\R) \implies \forall G\in
GL_n(\R), S\in SL_n(\R), GSG^{-1} \in$ $GL_n(\R)/SL_n(\R)$.

$\det: GL_n(\R)\twoheadrightarrow \R^*$ is a surjective homomorphism.

$\det(GSG^{-1}) = \det(G)\det(S)\det(G^{-1}) =
\det(G)\det(G)^{-1}\det(S) = \det(S) = 1$.

$\det(GS) = \det(G)\det(S) = \det(G) = r\in \R^*$.

So, each coset of $GL_n(\R)/SL_n(\R)$, corresponds to the equivalence
class of one $r\in\R^*$.

$\implies $$GL_n(\R)/SL_n(\R)\simeq \R^* \quad \blacksquare$

\paragraph{2} Let $G$ be a simple group and suppose $\phi$ $: G → G'$
is a homomorphism. Prove that $\phi$ is
either the trivial homomorphism or a one-to-one map.

\uwave{pf.}

Ker$(\phi) \trianglelefteq G$ and $G$ is a simple group.

Note that Ker$(\phi)$ is always normal. Since $G$ is simple the only normal subgroups of
$G$ are $\{e\}$ and $G$.

If Ker$\phi = G$ then $\phi^{-1}[\{e'\}] = G$, so $\phi$ is the
trivial homomorphism $G\rightarrow G'$.

If Ker$\phi = \{e\}$ then $\phi^{-1}[\{e'\}] = \{e\}$, so $\phi$ is
injective $\quad \blacksquare$

\paragraph{3} Let $G$ be a group and $N$ a normal subgroup of $G$. Provide a bijection between the set of
subgroups of $G/N$ and the set of subgroups of $G$ that contain $N$ (i.e. all $H ≤ G$ such that
$N ⊆ H$). Prove that it is indeed a bijection.

\uwave{pf. }

By hint, consider  $\pi: N \subseteq H_\alpha\leq G = S\rightarrow
\cup_{\alpha \in \Alpha} H_\alpha/N; H\mapsto H/N$

$\pi[S] = \cup_{\alpha \in \Alpha} H_\alpha/N$, since for each $H_\beta/N$, there is a $H_\beta$ that gets mapped
to $H_\beta/N$.

$\{e\}= N/N$ is the identity element of $\cup H_\alpha/N$.

Ker$(\pi)=\pi^{-1}[\{e\}] = \{H\in S| \pi(H) =\{e\}\}$

Let $H\in S,$ then $\pi(H) = H/N = \{e\}  \iff H = N$

$\implies$ Ker$(\pi) = N$

So, $\pi$ is a bijection $\quad \blacksquare$

\end{document}


%%% Local Variables:
%%% mode: latex
%%% TeX-master: t
%%% End:
